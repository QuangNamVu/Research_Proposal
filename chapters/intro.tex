\chapter{Giới thiệu} 
\section{Giới thiệu đề tài}
(Logistic Regression), cây hồi quy và phân loại (CART),
rừng ngẫu nhiên (Random Forest),
mạng nơ-ron (Neural NetWork),
máy véctơ hỗ trợ (SVM).
Nhưng ở Việt Nam lại chưa có nhiều nghiên cứu về đề tài này.
Vậy nên tôi quyết định chọn đề tài \textbf{Dự đoán xu hướng giá ngắn hạn các đồng tiền mật mã bằng kĩ thuật học máy}.
\section{Mục tiêu và phạm vi đề tài}
\subsection{Mục tiêu}
Mục tiêu của luận văn này là xây dựng một công cụ dự đoán xu hướng giá ngắn hạn
các đồng tiền mật mã bằng kĩ thuật học máy. Dữ liệu đầu vào là các thông tin về
lịch sử giá các  đồng tiền ảo trong các phiên giao dịch.
\subsection{Phạm vi đề tài}
\begin{itemize}
\item Tìm hiểu và nghiên cứu về lý thuyết học máy thống kê (statistical machine learning)
\item Xây dựng mô hình dự đoán vế xu hướng tăng giảm, dự đoán giá của các đồng trong thời gian ngắn hạn.
\end{itemize}
Các đối tượng nghiên cứu trong đề tài:
\begin{itemize}
\item Các tài liệu liên quan tới lý thuyết thống kê hiện đại
\item Các mô hình trong học máy: hồi quy logistic, rừng ngẫu nhiên, mạng nơ-ron
\item Sử dụng ngôn ngữ Python, R và một số thư viện để hiện thực mô hình.
\end{itemize}

\section{Tiến độ thực hiện}
Trong phần này, tác giả xin trình bày lịch trình công việc đã thực hiện đề tài trong học
kỳ I và lịch trình dự kiến hiện thực đề tài trong quá trình làm luận văn chính thức ở học
kỳ II dưói dạng biểu đồ Gantt sau đây.

% \documentclass[tikz]{standalone}
% \usepackage{pgfgantt}
\title{Biểu đồ kế hoạch}

\begin{ganttchart}[
    canvas/.append style={fill=none, draw=black!15, line width=.75pt},
    hgrid style/.style={draw=black!15, line width=.75pt},
    vgrid={*1{draw=black!15, line width=.5pt}},
    x unit=.6cm,
    %y unit title=0.7cm,
    y unit chart=0.65cm,
    today=15,
    today rule/.style={
      draw=black!64,
      dash pattern=on 3.5pt off 4.5pt,
      line width=1.5pt
    },
    today label font=\small\bfseries,
    title/.style={draw=none, fill=none},
    title label font=\bfseries\footnotesize,
    title label node/.append style={below=7pt},
    include title in canvas=false,
    bar label font=\mdseries\small\color{black!70}, %text color%
    bar label node/.append style={left=.1cm},
    bar/.append style={draw=none, fill=black!30},
    % bar incomplete/.append style={fill=barblue},
    bar height=.4,
    % bar progress label font=\mdseries\footnotesize\color{black!65},
    group/.append style={draw=none, fill=black!60},
    % group incomplete/.append style={fill=groupblue},
    group left shift=0,
    group right shift=0,
    group height=.3,
    group peaks tip position=0,
    group label node/.append style={left=.1cm},
  ]{1}{15}
  \gantttitle[
    title label node/.append style={below left=7pt and -3pt}
  ]{Tuần:\quad1}{1}
  \gantttitlelist{2,...,15}{1} \\
  \ganttgroup[progress label text={}]{Học kì 1}{1}{15} \\
  \ganttbar[
    progress label text={}
  ]{Hiện thực việc thu thập dữ liệu một cách tự động}{1}{15} \\
  \ganttbar[
  progress label text={}
  ]{Nghiên cứu các tính chất về chuỗi thời gian}{1}{5} \\
  
  \ganttbar[
    progress label text={}
  ]{Nghiên cứu các khái niệm về Rừng ngẫu nhiên}{3}{7} \\
  \ganttbar[
    progress label text={}
  ]{Nghiên cứu các thành phần trong mạng nơron tích chập}{6}{12} \\
  \ganttbar[
    progress label text={}
  ]{Đánh giá kết quả sử dụng mạng nơron tích chập}{8}{14} \\
	\ganttbar[
	progress label text={}
	]{Nghiên cứu mô hình Markov ẩn}{13}{15} \\
  \ganttbar[
    progress label text={}
  ]{Báo cáo đề cương}{13}{14}
  
%  \gantbar[]
  %\ganttbar[
  %  progress label text={}
 % ]{Thuyết trình đề cương}{15}{15}
 \\[grid]

  %-----------------------------------------------------

  \ganttgroup[progress label text={}]{Học kì 2}{1}{15} \\
  \ganttbar[
    progress label text={}
  ]{Nghiên cứu sử dụng mô hình Markov ẩn}{1}{5} \\
  \ganttbar[
    progress label text={}
  ]{Nghiên cứu các thành phần trong mạng nơron hồi quy}{6}{8} \\
  \ganttbar[
  progress label text={}
  ]{Nghiên cứu các thành phần trong mạng bộ nhớ dài-ngắn}{8}{9} \\
  \ganttbar[
    progress label text={}
  ]{Đánh giá các mô hình}{6}{14} \\
  \ganttbar[
    progress label text={}
  ]{Báo cáo luận văn}{14}{14} \\
  \ganttbar[
    progress label text={}
  ]{Thuyết trình đề cương}{15}{15}
\end{ganttchart}