\chapter{Giới thiệu} 
\section{Giới thiệu đề tài}
%Quản lý bệnh viện là một phạm trù đòi hỏi người quản lý phải có cách nhìn tổng quan, bao quát cả về môi trường ngành Y tế cũng như các nhân tố tác động trực tiếp đến hệ thống quản lý bệnh viện. Trong thời buổi kinh tế ngày một phát triển, hoạt động kinh tế đang nóng từng ngày thì vấn đề thời gian đang rất được người dân quan tâm. Các cơ sở y tế đang phải đối mặt với vấn đề thiếu hụt các nhân viên y tế đồng thời phải tăng chất lượng dịch vụ, đáp ứng nhu cầu của bệnh nhân không chỉ về mặt sức khỏe mà còn sự hài lòng của bệnh nhân. Thực hiện việc quản lý hiệu quả sẽ giúp cải thiện chất lượng bệnh viện như chống quá tải, sử dụng hiệu quả cơ sở hạ tầng, phòng chống rủi ro tài chính. Bệnh nhân cũng sẽ được hưởng dịch vụ tốt hơn với giá cả phải chăng. Trong đó, thời gian điều trị và số lần tái nhập viện đóng một vai trò quan trọng trong việc quản lý nguồn tài nguyên của bệnh viện.\\
%Dự đoán chính xác thời gian điều trị cho phép bệnh viện xác định được ngày xuất viện với độ tin cậy cao. Việc này có thể giúp quá trình xuất nhập viện linh động hơn, chủ động trong việc bố trí giường bệnh, giúp dự đoán chi phí điều cho trị bệnh nhân. Xa hơn nữa, nó giúp các nhà quản lý vạch ra chiến lược phát triển lâu dài đúng đắn cho bệnh viện. Tuy nhiên, việc tính toán, xác định thời gian điều trị cho từng bệnh nhân không phải là một công việc đơn giản, chưa tính đến bệnh nhân có nhiều đợt xuất nhập viện và tình hình bệnh của người đó.\\
Hiện nay trên thế giới đã có các mô hình dự đoán sử dụng hồi quy logistic
(Logistic Regression), cây hồi quy và phân loại (CART),
rừng ngẫu nhiên (Random Forest),
mạng nơ-ron (Neural NetWork),
máy véctơ hỗ trợ (SVM).
Nhưng ở Việt Nam lại chưa có nhiều nghiên cứu về đề tài này.
Vậy nên tôi quyết định chọn đề tài \textbf{Dự đoán xu hướng giá ngắn hạn các đồng tiền mật mã bằng kĩ thuật học máy}.
\section{Mục tiêu và phạm vi đề tài}
\subsection{Mục tiêu}
Mục tiêu của luận văn này là xây dựng một công cụ dự đoán xu hướng giá ngắn hạn
các đồng tiền mật mã bằng kĩ thuật học máy. Dữ liệu đầu vào là các thông tin về
lịch sử giá (TODO:các)  đồng tiền ảo trong các phiên giao dịch.
\subsection{Phạm vi đề tài}
\begin{itemize}
\item Tìm hiểu và nghiên cứu về lý thuyết học máy thống kê (statistical machine learning)
\item Xây dựng mô hình dự đoán vế xu hướng tăng giảm, dự đoán giá của các đồng trong thời gian ngắn hạn.
\end{itemize}
Các đối tượng nghiên cứu trong đề tài:
\begin{itemize}
\item Các tài liệu Toán ma trận.
\item Các tài liệu liên quan tới lý thuyết thống kê hiện đại.
\item Các mô hình trong học máy: hồi quy logistic, rừng ngẫu nhiên.
\item Các mô hình trong học sâu: các mô hình mạng neural: CNN, RNN, LSTM.
\item Sử dụng ngôn ngữ Python, R và một số thư viện để hiện thực mô hình.
\end{itemize}
% \begin{table}[h!]
%     \centering
%     \begin{tabular}{|M{2.5cm}|M{2.5cm}|M{9cm}|}
%     \hline
%     Hàm truyền & Công thức & Đồ thị \\
%     \hline
%     Linear & $f(x) = x$ & \resizebox{.5\textwidth}{!}{
% \begin{tikzpicture}[domain=-1:1,yscale = 2,xscale = 4,smooth]
% \datavisualization [school book axes,
%                     visualize as smooth line,
%                     y axis={label={$f(x)$}},
%                     x axis={label} ]

% data [format=function] {
%       var x : interval [-1.5:1.5] samples 7;
%       func y = \value x;
%       };
% \end{tikzpicture}
% } \\
% \hline
% Sigmoid & $f(x) = \frac{1}{1+e^{-x}}$ & \resizebox{.5\textwidth}{!}{
% \begin{tikzpicture}[domain=-6:6,yscale = 1,xscale = 1,smooth]
% \datavisualization [school book axes,
%                     visualize as smooth line,
%                     y axis={label={$f(x)$}},
%                     x axis={label} ]

% data [format=function] {
%       var x : interval [-6:6] samples 7;
%       func y = 1/(1+exp(- \value x));
%       };
% \draw[color=black, line width = 0.3pt, domain=-6:6, samples=7,smooth]     plot (\x,{1});
% \end{tikzpicture}
% }\\
% \hline
% Tan-Sigmoid & $f(x) = \frac{1-e^x}{1+e^{x}}$ & \resizebox{.5\textwidth}{!}{
% \begin{tikzpicture}[domain=-6:6,yscale = 1,xscale = 1,smooth]
% \datavisualization [school book axes,
%                     visualize as smooth line,
%                     y axis={label={$f(x)$}},
%                     x axis={label} ]

% data [format=function] {
%       var x : interval [-6:6] samples 7;
%       func y = - (1 - exp(\value x))/(1 + exp(\value x)));
%       };
% \draw[color=black, line width = 0.3pt, domain=-6:6, samples=7,smooth]     plot (\x,{1});
% \draw[color=black, line width = 0.3pt, domain=-6:6, samples=7,smooth]     plot (\x,{-1});
% \end{tikzpicture}
% }\\
% \hline
% ReLu & $f(x) = \max (0,x)$ & \resizebox{.5\textwidth}{!}{
% \begin{tikzpicture}[domain=-6:6,yscale = 1,xscale = 1.5]
% \datavisualization [school book axes,
%                     visualize as smooth line,
%                     y axis={label={$f(x)$}},
%                     x axis={label} ]

% data [format=function] {
%       var x : interval [-4:4] samples 300;
%       func y = max(0,\value x);
%       };
% \end{tikzpicture}
% }\\
% \hline
%     \end{tabular}
%      \caption{Các hàm truyền trong CNN \cite{softmax} }
%     \label{refhinh6}
% \end{table}

