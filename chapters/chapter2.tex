\chapter{Tổng quan về lĩnh vực nghiên cứu} 
\section{Những yếu tố tác động đến giá trị đồng tiền mã hóa}
\subsection{Cung và cầu của thị trường}
Trong nguyên tắc chính của kinh tế nếu người ta mua một đồng tiền, giá trị của đồng tiền sẽ tăng lên và nếu người ta bán đồng tiền, giá sẽ giảm.
\subsection{Tin tức trên các phương tiện thông tin đại chúng}
Các sự kiện chính trị và kinh tế trên toàn thế giới ảnh hưởng đến cách mà con người phản ứng với các dự đoán giá, tin tức cảnh báo về rủi ro tác động chính lên cung-cầu.
\subsection{Quy định của chính phủ}
Có 4 cấp độ quản lý tiền ảo hiện nay đang được các nước thực thi, cụ thể:
\begin{itemize}
\item Cấm trên diện rộng.
\item Cấm trong lĩnh vực tài chính ngân hàng (trong đó có Trung Quốc).
\item Cảnh báo rủi ro đối với người sử dụng, đầu tư,.
\item  Chấp nhận như một phương tiện thanh toán (trong đó có Hàn Quốc, Nhật Bản và Mỹ).
\end{itemize}
cập nhật ngày 14/4/2018.
\subsection{Chính sách của các tổ chức}
Facebook, Google và Twitter đã ngăn chặn khách hàng và người dùng sử dụng dịch vụ cryptocurrency.
\subsection{Các vấn đề kỹ thuật}
Vì đồng tiền mã hóa có thể bị hack thành công vào tài khoản hoặc tấn công máy chủ, có thể làm giảm tỷ giá hối đoái, dẫn đến giá giảm.
\section{Nhu cầu sử dụng tiền mã hoá của mỗi hệ sinh thái}
\begin{itemize}
    \item Số thành viên tham gia vào hệ sinh thái (Số người đến khu vui chơi mua vé tham gia các trò chơi trong đó bằng tiền A).
    \item Số lượng dịch vụ trong hệ sinh thái (Khu vui chơi có càng nhiều trò chơi thì nhu cầu sử dụng tiền A càng tăng); Và các nền tảng như Ethereum luôn mở cho các đối tác tạo các dịch vụ gia tăng trên đó giống như khu vui chơi cho phép đối tác bên ngoài vào tổ chức trò chơi ở trong.
    \item  Số người đầu cơ: Những người nhận thấy nhu cầu tiền mã hoá của một hệ sinh thái tăng dần sẽ mua để nắm giữ chờ tăng giá thì bán ra. (Giống như phe vé bóng đá ngày trước mua vé chờ sát trận nhu cầu tăng vọt thì bán ra. Khu vui chơi thì ít có nhóm này vì lượng vé không bị giới hạn).
    \item  Số người bán bên ngoài chấp nhận tiền mã hoá: Một số người bán nhận thấy tính thanh khoản của tiền mã hoá và giá trị tăng dần của nó nên đã chấp nhận khách hàng thanh toán các hàng hoá dịch vụ của mình bằng loại tiền này (Nhà hàng bên cạnh khu vui chơi có thể chấp nhận khách hàng thanh toán bằng tiền A).
\end{itemize}
% \section{Các công trình liên quan}
% LOS là một thước đo quan trọng về hiệu quả chăm sóc sức khỏe. Nhưng nó phải đi đôi với chất lượng.Các áp lực tài chính trong việc giảm các khoản chi trả hoặc và gói thanh toán đi kèm thúc đẩy việc giảm LOS \cite{1}. Ngược lại, LOS dài hơn cho phép cải thiện việc chăm sóc tập trung cho bệnh nhân và chuyển bệnh nhân sang các cơ sở chăm sóc sức khỏe, điều này có thể làm giảm tỷ lệ tái nhập viện và cải thiện sức khỏe khi xuất viện \cite{2}. Swaminathan và các cộng sự đã nghiên cứu xem liệu việc giảm LOS đối với các bệnh nhân lớn tuổi trải qua phẫu thuật can thiệp mạch vành qua da do bị nhồi máu cơ tim xuyên thành có an toàn hay không\cite{16}.\\
% Dựa trên các dấu hiệu bệnh của bệnh nhân và các kết quả của phòng xét nghiệm vào thời điểm nhập viện, họ kết luận rằng các bệnh nhân có LOS ngắn ($<3$ ngày) có kết quả phẫu thuật tốt hơn. Họ cũng nhận thấy rằng các bệnh nhân có LOS dài thường là người cao tuổi và cũng mắc nhiều bệnh kèm theo hơn nhóm có LOS ngắn và trung bình. Một nghiên cứu khác của Whellan và các đồng sự của mình trên các bệnh nhân bị suy tim \cite{17} cũng chỉ ra điều tương tự.\\ Tuy nhiên các bệnh nhân có LOS ngắn lại có tỉ lệ tái nhập và tỉ lệ tử vong sau xuất viện cao hơn\cite{3}. Tuy chưa có bằng chứng cụ thể về quan hệ giữa thời gian nằm viện ngắn và tỉ lệ tái nhập viện/tử vong nhưng ta có thể phần nào thấy được hiệu suất chăm sóc sức khỏe thật sự đối vối bệnh nhân.\\
% Trên thực tế, LOS thường có độ lệch cao, dù vậy vẫn có những nghiên cứu phân tích LOS trung bình và sử dụng các mô hình dùng cho dữ liệu có độ lệch thấp. Hoặc có một số nghiên cứu bỏ qua việc kiểm tra sai lệch trước khi bắt đầu nghiên cứu của họ, sử dụng mô hình hồi quy tuyến tính \cite{4} \cite{5}, hoặc sử dụng các kỹ thuật để chống lại vấn đề dữ liệu phân tán, ví dụ như Jonas và đồng sự sử dụng hồi quy logistic chia dữ liệu thành hai nhóm LOS ngắn($\leq 10$ ngày) và LOS dài ($> 10$ ngày) \cite{6}.\\
% Các nghiên cứu tiếp theo sử dụng các mô hình phù hợp hơn. Verburg dự đoán LOS trong một đơn vị chăm sóc đặc biệt (ICU) sử dụng các mô hình hồi quy gồm: hồi quy bình phương nhỏ nhất, mô hình tuyến tính tổng quát với phân phối chuẩn và một hàm liên kết logarit, hồi quy Poisson, hồi quy nhị thức âm, hồi quy Gamma với một hàm liên kết logarit, hồi quy Cox, APACHE IV. Tuy nhiên hiệu quả dự đoán của các mô hình này vẫn còn thấp \cite{9}.\\
% Các thuật toán học máy tỏ ra là phù hợp hơn cả để mô hình LOS, Houthooft và các cộng sự so sánh các thuật toán học máy khác nhau gồm: support vector machines (SVMs), support vector regression (SVR), relevance vector machines (RVMs), artificial neural networks (ANNs), k-nearest neighbours (k-NN), classification and regression trees (CART), random forests (RF), discrete adaptive boosting và feature selection để dự đoán LOS của 14480 bệnh nhân ICU. Mô hình hiệu quả nhất là SVR, với sai số tuyệt đối trung bình là $1.79$ ngày đối với các bệnh nhân có khoảng thời gian sống sót không lâu ($< 10$ ngày) \cite{7}.\\
% Trong một nghiên cứu gần đây, Lior và các đồng sự sử dụng mô hình Cubist để dự đoán LOS của hơn 4800 bệnh nhân bị suy tim sung huyết tại thời điểm nhập viện. Mô hình sử dụng thuật toán CARMA để gom nhóm các đặc tính liên kết và sử dụng SVM để phân mức sai số dự đoán của Cubist \cite{10}, nghiên cứu cũng so sánh kết quả với một số mô hình khác như ANNs, CART, cây phát hiện sự tương tác thuộc tính mục tiêu (CHAID), hồi quy Poisson, SVM. Kết quả thu được là hết sức khả quan với sai số trung bình tuyệt đối là $1$ ngày đối với các bệnh nhân có thời gian nằm viện dưới 31 ngày.\\
% Trong đề tài này, chúng tôi sẽ xây dựng một mô hình dự đoán LOS theo như \cite{10} và tiếp tục cải tiến, bổ sung thêm những mô hình mới khả thi.